\documentclass[11pt,spanish]{article}

\input{preamble.tex} 

\newcommand{\tnum}{2 y 3}
\newcommand{\sem}{2024-2}
\newcommand{\campus}{San Joaquín \\ Santiago}
\newcommand{\rolusm}{202273531-7}
\newcommand{\namestudent}{Diego Sierra}

\headheight=14pt
\linespread{1.3}
\author{\namestudent}
\pagestyle{fancy}
\fancyhf{}%
\fancyfoot[R]{ \namestudent \\ \rolusm}
\fancyfoot[L]{Campus \campus} 
\fancyfoot[C]{\thepage}
\rhead{2024-2}
\lhead{INF-221}
\renewcommand{\headrulewidth}{0.4pt}
\renewcommand{\footrulewidth}{0.4pt}
\newbool{programs}
\boolfalse{programs}
\chead{INFORME TAREA \tnum~}



\title{
  \huge
  \textbf{INFORME TAREA \tnum~ \\ ALGORITMOS Y COMPLEJIDAD} \\[1ex]
  \emph{\textquote{Explorando la Distancia entre Cadenas, una Operación a la Vez}}
  }

  
\date{
  \small
  \today
}




\begin{document}
\maketitle
\thispagestyle{fancy} 
\vspace{-1.0\baselineskip}




\begin{abstract}
  \textit{ 
    Este informe explora algoritmos para calcular la distancia de edidicion extendida (agregando transposicion) implementando 2 algoritmos,
fuerza bruta y programación dinámica, comparando sus eficiencias reales y enfatizando la importancia de seleccionar el algoritmo
correcto segun las necesidades del problema a resolver, se exploran las complejidades computacionales, estrategias de resolución y 
rendimiento, se estudian estos algoritmos con diferentes datasets, las operaciones ocupadas son inserciones, eliminaciones, 
sustituciones y transposiciones.
  }
     
\end{abstract}

\setcounter{tocdepth}{1}
\tableofcontents


\newpage
\section{Introducción}
El problema de edicion de caracteres es un clasico problema en ciencias de la computacion, en el cual se busca encontrar 
la cantidad minima de operaciones para transformar una cadena de caracteres en otra. Este problema es de gran importancia en la vida cotidiana,
ya que se puede aplicar en la corrección de errores de texto, en la comparación de archivos, analisis de similitud de texto, etc.


Este informe esta enfocado en la implementación de 2 algoritmos para la resolución de estos problemas. En particular, 
se busca encontrar la solucion de, dada 2 cadenas de caracteres cual es el costo minimo de edicion de 1 de estas cadenas para que las cadenas sean iguales.
La investigacion de centra en analizar los distintos algoritmos para poder determinar cual es el mas eficiente en terminos de tiempo y espacio. Acaso
programación dinámica es mejor que fuerza bruta en todos los casos ? o hay casos en los que fuerza bruta es mejor que programación dinámica ?
Estas son algunas de las preguntas que se buscan responder con esta investigacion, la respuesta de estas preguntas es de gran importancia ya que
nos permitira saber como se comportan estos algoritmos en diferentes situaciones y asi poder elegir el algoritmo mas adecuado para cada caso.


Los algoritmos a implementar son, de programación dinámica y de fuerza bruta.


Las operaciones permitidas para este estudio seran: inserción, eliminación, sustitucion y transposicion de caracteres. Para la implementación 
de los algoritmos se utilizó el lenguaje de programación de C++.


Es importante que tengan a mano el codigo fuente de este estudio para poder seguir la explicación de los algoritmos, poder entender
como se implementan, como funcionan y por si se les presenta dudas sobre codigo en particular.
\url{https://github.com/Daspssj/Tarea2_3}


\newpage
\section{Diseño y Análisis de Algoritmos} 
\input{sections/design_analysis}

\subsection{Fuerza Bruta}
La solucion diseñada para el algoritmo de fuerza bruta se enfoca en recorrer todas las posibles combinaciones
de las operaciones de inserción, eliminacion, sustitución y transposición, comparando la cadena de strings 
\texttt{S1} y \texttt{S2} para encontrar la distancia minima de edicción extendida, modificando unicamente una
de estas 2 cadenas en cada iteración, en nuestro caso solo se modifica la primera cadena \texttt{S1}.

Como hay que expresar las complajidades en terminos de las longitudes de las cadenas de entrada
$S1$ y $S2$, se tiene que n y m son las longitudes de las cadenas $S1$ y $S2$ respectivamente
y se tiene que la complejidad temporal de este algoritmo es de $O(4^{\max(n,m)})$ y la complejidad espacial
es de $O(\max(n,m))$.


La transposiciones y costos variables impactan significativamente en la complejidad en el caso de las transposiciones
se tiene que la complejidad temporal aumenta a $O(4^{\max(n,m)})$ siendo la complejidad temporal sin la transposición
de $O(3^{\max(n,m)})$ esto se debe a que en el peor caso se tiene una mayor cantidad de llamadas recursivas, en cambio 
la complejidad espacial con la transposición es la misma que la vista anteriormente.


Para el ejemplo de ejecucionse tienen las cadenas $S1$ = "kitten" y $S2$ = "sitting", se tiene que la distancia de edición mínima es de 1,
y se puede obtener de la siguiente manera:

\begin{itemize}
    \item \textbf{Sustituir 'k' por 's':} k \textrightarrow s
    \item \textbf{Insertar 'g' al final de la cadena:} \textrightarrow g
    \item \textbf{Sustituir 'e' por 'i':} e \textrightarrow i
    
\end{itemize}

\begin{algorithm}[H]
    \SetKwProg{myproc}{Procedure}{}{End}
    \SetKwFunction{FuerzaBruta}{FuerzaBruta}  
    \SetKwFunction{CostoSust}{CostoSust}
    \SetKwFunction{CostoElim}{CostoElim}
    \SetKwFunction{CostoInser}{CostoInser}
    \SetKwFunction{CostoTrans}{CostoTrans}

    \DontPrintSemicolon
    \footnotesize

    % Definición del algoritmo principal
    \myproc{\FuerzaBruta{S1, S2}}{
        $n \leftarrow$ longitud de S1\;
        $m \leftarrow$ longitud de S2\;
        
        \uIf{$n = 0$}{
            \Return costo de insertar todos los caracteres de S2\;
        }
        \uElseIf{$m = 0$}{
            \Return costo de eliminar todos los caracteres de S1\;
        }
        
        $costo\_sust \leftarrow$ \CostoSust{S1[n-1], S2[m-1]} + \FuerzaBruta{S1[0:n-1], S2[0:m-1]}\;
        $costo\_elim \leftarrow$ \CostoElim{S1[n-1]} + \FuerzaBruta{S1[0:n-1], S2}\;
        $costo\_inser \leftarrow$ \CostoInser{S2[m-1]} + \FuerzaBruta{S1, S2[0:m-1]}\;
        $costo\_trans \leftarrow \infty$\;

        \uIf{$n > 1$ \textbf{and} $m > 1$ \textbf{and} $S1[n-1] = S2[m-2]$ \textbf{and} $S1[n-2] = S2[m-1]$}{
            $costo\_trans \leftarrow$ \CostoTrans{S1[n-2], S1[n-1]} + \FuerzaBruta{S1[0:n-2], S2[0:m-2]}\;
        }
        
        \Return $\min(costo\_sust, costo\_elim, costo\_inser, costo\_trans)$\;
    }
    \caption{Fuerza Bruta para calcular la distancia mínima de edición.}
    \label{alg:fuerza_bruta}
\end{algorithm}

\subsection{Programación Dinámica}




\subsubsection{Descripción de la solución recursiva}

Este algoritmo busca calcular el costo mínimo de transformar una cadena de caracteres $S1$ en otra 
cadena de caracteres $S2$. Para ello, se consideran las siguientes operaciones:
\begin{itemize}
    \item \textbf{Inserción:} Insertar un carácter a la cadena $S1$.
    \item \textbf{Eliminación:} Eliminar un carácter de la cadena $S1$.
    \item \textbf{Sustitución:} Reemplazar un carácter de la cadena $S1$ por otro que le correspon a $S2$.
    \item \textbf{Transposición:} Intercambiar dos caracteres consecutivos de la cadena $S1$ y $S2$.
\end{itemize}


Se enfoca en comparar el costo de realizar cada una de estas operaciones para cada caracter para asi 
elejir la operación que minimice el costo total de transformar $S1$ en $S2$.

\subsubsection{Relación de recurrencia}

Sea dp[i][j] el costo mínimo de transformar los primeros i caracteres de S1 en los primeros j caracteres de S2, 
tenemos que la relación de recurrencia es la siguiente:

\[
dp[i][j] =
\begin{cases} 
    j \cdot \text{costo\_inser}(s2[j-1]), & \text{si } i = 0 \\
    i \cdot \text{costo\_elim}(s1[i-1]), & \text{si } j = 0 \\
    \min \bigg(
        dp[i][j-1] + \text{costo\_inser}(s2[j-1]), & \text{(insertar)} \\
        dp[i-1][j] + \text{costo\_elim}(s1[i-1]), & \text{(eliminar)} \\
        dp[i-1][j-1] + \text{costo\_sust}(s1[i-1], s2[j-1]), & \text{(sustituir)} \\
        dp[i-2][j-2] + \text{costo\_transpos}(s1[i-1], s2[j-1]) & \text{(transponer, si aplica)} 
    \bigg)
\end{cases}
\]

Como se puede obeservar en la relación de recurrencia, se consideran los casos base cuando $i = 0$ y $j = 0$,
que corresponden a los costos de insertar y eliminar todos los caracteres de $S2$ y $S1$, respectivamente, y tambien
se incluye la operación de transposición si se cumplen las condiciones necesarias.

\subsubsection{Identificación de subproblemas}

La soluicon que propone este algoritmo de transformar una cadena de caracteres $S1$ en otra cadena de caracteres $S2$ 
(entregando los costos asociados) se puede dividir en subproblemas más pequeños, que corresponden a transformar
subcadenas de $S1$ en subcadenas de $S2$, se busca transformar los prefijos inmediatos de $S1$ y $S2$ en cada iteración
para luego sumar el costo de la operación que minimice el costo total de transformar $S1$ en $S2$.

\subsubsection{Estructura de datos y orden de cálculo}

Para resolver este problema utilizando programación dinámica, se propone utilizar una matriz $dp$ de tamaño $(n+1) 
\times (m+1)$,donde $n$ y $m$ son las longitudes de las cadenas $S1$ y $S2$, respectivamente. El programa utiliza un
programación dinamica don un enfoque de Bottom-Up. En la matriz se inicializan los valores de los casos base de
manera que la primera fila es el costo de insertar caracteres en $S1$ y la primera columna respresenta el costo de
eliminar caracteres de $S1$,luego se recorren las filas y columnas de la matriz para calcular el costo mínimo 
de transformar los prefijos de $S1$ y $S2$ en cada iteración, se llenan los valores de la matriz de izquierda a derecha
y de abajo hacia arriba, de manera que al final de la ejecución el valor de $dp[n][m]$ corresponderá al costo mínimo.

\subsubsection{Complejidades}

La complejidad temporal y espacial de este algoritmo son las mismas y es de $O(n \times m)$, donde $n$ y $m$ son 
las longitudes de las cadenas $S1$ y $S2$, respectivamente.

\subsubsection{Ejemplo de ejecución}

Para las cadenas $S1$ = "kitten" y $S2$ = "sitting", se tiene que la distancia de edición mínima es de 1,
y se puede obtener de la siguiente manera:

\begin{itemize}
    \item \textbf{Sustituir 'k' por 's':} k \textrightarrow s
    \item \textbf{Insertar 'g' al final de la cadena:} \textrightarrow g
    \item \textbf{Sustituir 'e' por 'i':} e \textrightarrow i
    
\end{itemize}

\subsubsection{Algoritmo utilizando programación dinámica}

\begin{algorithm}[H]
    \SetKwProg{myproc}{Procedure}{}{End}
    \SetKwFunction{ProgDinamica}{ProgDinamica}  
    \SetKwFunction{CostoSust}{CostoSust}
    \SetKwFunction{CostoElim}{CostoElim}
    \SetKwFunction{CostoInser}{CostoInser}
    \SetKwFunction{CostoTrans}{CostoTrans}

    \DontPrintSemicolon
    \footnotesize

    \myproc{\ProgDinamica{S1, S2}}{
        $n \leftarrow$ longitud de S1\;
        $m \leftarrow$ longitud de S2\;
        Crear matriz $dp$ de tamaño $(n+1) \times (m+1)$\;

        \For{$i \leftarrow 0$ \textbf{to} $n$}{
            $dp[i][0] \leftarrow$ costo de eliminar todos los caracteres hasta $i$ en S1\;
        }
        \For{$j \leftarrow 0$ \textbf{to} $m$}{
            $dp[0][j] \leftarrow$ costo de insertar todos los caracteres hasta $j$ en S2\;
        }

        \For{$i \leftarrow 1$ \textbf{to} $n$}{
            \For{$j \leftarrow 1$ \textbf{to} $m$}{
                $costo\_inser \leftarrow dp[i][j-1] + \CostoInser{S2[j-1]}$\;
                $costo\_elim \leftarrow dp[i-1][j] + \CostoElim{S1[i-1]}$\;
                $costo\_sust \leftarrow dp[i-1][j-1] + \CostoSust{S1[i-1], S2[j-1]}$\;
                $costo\_trans \leftarrow \infty$\;
                
                \uIf{$i > 1$ \textbf{and} $j > 1$ \textbf{and} $S1[i-1] = S2[j-2]$ \textbf{and} $S1[i-2] = S2[j-1]$}{
                    $costo\_trans \leftarrow dp[i-2][j-2] + \CostoTrans{S1[i-2], S1[i-1]}$\;
                }

                $dp[i][j] \leftarrow \min(costo\_inser, costo\_elim, costo\_sust, costo\_trans)$\;
            }
        }

        \Return $dp[n][m]$\;
    }
    \caption{Programación Dinámica para calcular la distancia mínima de edición.}
    \label{alg:prog_dinamica}
\end{algorithm}

\newpage
\section{Implementaciones}
\begin{itemize}
    \item \textbf{Estructura de archivos:} El archivo donde se encuentra la implementación de los algoritmos es
    Algoritmos.cpp , en este archivo se encuentran las implementaciones de los algoritmos de fuerza bruta y programación dinámica,
    como la funcion del main que se encarga de leer los datos de entrada y llamar a los algoritmos correspondientes.


    En esta carpeta de la Tarea 2 y 3 tambien se encuentran los costos de las operaciones de inserción, eliminación, 
    sustitución y transposición, como tambien las distintas datasets que se utilizaron para probar los algoritmos.


    Tambien se encuentra el archivo de Creardataset.cpp que se encarga de crear los datasets que se utilizaron para probar
    los algoritmos.
    \item \textbf{Funciones:} Las funciones mas importantes son dynamic progra y brute force, que corresponden a 
    las implementaciones de los algoritmos de programación dinámica y fuerza bruta, respectivamente.
\end{itemize}


\newpage
\section{Experimentos}

\subsection{Infraestructura utilizada}

Para la realización de los experimentos, se utilizó un computador con las siguientes características:

\begin{itemize}
    \item \textbf{Procesador:} Intel Core i7-14700KF, 5.6 GHz
    \item \textbf{Memoria RAM:} 32 GB DDR5
    \item \textbf{Almacenamiento:} 2 TB SSD NVMe M.2
    \item \textbf{Sistema Operativo:} Windows 10 Pro
    \item \textbf{Compilador:} g++ 11.4.0
    \item \textbf{Modo de compilacion} g++ -Wall \{Archivo\}.cpp -o test
    \item \textbf{Librerías:} C++ Standard Library
    \item \textbf{Entorno de Desarrollo:} Ubuntu 11.4.0 
\end{itemize}


\subsection{Datasets}
Los datasets utilizados para las pruebas fueron 6, y son de 32 palabras de largo variablem mas no aleatorio, estos datasets
fueron generados de menera automatica y se guardaron en archivos de texto, cada dataset tiene un nombre, y fueron las siguientes:
\begin{itemize}
    \item \textbf{dataset\_mismotamano.txt:} Contiene palabras de misma longitud que varia entra 1 y 15 caracteres.
    \item \textbf{dataset\_s1\_vacia.txt:} Contiene palabras donde S1 es vacia y S2 tiene longitud variable.
    \item \textbf{dataset\_s2\_vacia.txt:} Contiene palabras donde S2 es vacia y S1 tiene longitud variable.
    \item \textbf{dataset\_s1>s2.txt:} Contiene palabras donde S1 es mayor o igual que S2.
    \item \textbf{dataset\_s1<s2.txt:} Contiene palabras donde S2 es mayor o igual que S1.
    \item \textbf{dataset\_transposicion.txt:} Contiene palabras transpuestas.
\end{itemize}

La importancia de estos datasets radica en que se pueden probar los algoritmos con distintos casos de prueba, y se pueden
verificar si los algoritmos son capaces de resolverlos de manera correcta en varios casos, esto nos puede dar una idea de
la eficiencia y efectividad de los algoritmos, lo que se busca probar con estos datasets es que los algoritmos sean capaces
de resolver problemas donde las cadenas sean de distinto e igual tamaño, que pueda resolver problemas donde los dos strings o solamente
uno de ellos este vacio y por ultimo que pueda resolver problemas donde las cadenas de caracteres esten transpuestas.

\subsection{Resultados}
Para el analisis de los resultados se tiene que hay que destacar que en el eje y estan los timepos de ejecucion en microsegundos
y en el eje x se tiene el largo de las cadenas de entrada, se puede observar que en la mayoria de los casos el tiempo de ejecucion
es muy bajo, esto se debe a que las cadenas de entrada son muy pequeñas, en el caso de las cadenas de entrada de longitud 14 se puede
observar que el tiempo de ejecucion es mayor, esto se debe a que la complejidad del algoritmo es de $O(4^{\max(n,m)})$ y la complejidad
espacial es de $O(\max(n,m))$.
Se puede observar tambien que en el caso de las cadenas de entradas para s1 y s2 vacios el tiempo de ejecucion de la fuerza bruta es mejor
que el de programacion dinamica.


\begin{figure}[H]
    \centering
    \begin{minipage}[t]{0.5\textwidth}
        \includegraphics[width=\textwidth]{images/mismotamanio.png}
    \end{minipage}%
    \begin{minipage}[t]{0.5\textwidth}
        \includegraphics[width=\textwidth]{images/transpuesto.png}   \end{minipage}%
    \caption{Tiempos de ejecucion vs largo de las cadenas}
    \label{fig:Tiempos de ejecucion vs largo de las cadenas}
\end{figure}

\begin{figure}[H]
    \centering
    \begin{minipage}[t]{0.5\textwidth}
        \includegraphics[width=\textwidth]{images/s1vacio.png}
    \end{minipage}%
    \begin{minipage}[t]{0.5\textwidth}
        \includegraphics[width=\textwidth]{images/s2vacio.png}   \end{minipage}%
    \caption{Tiempos de ejecucion vs largo de las cadenas}
    \label{fig:Tiempos de ejecucion vs largo de las cadenas}
\end{figure}

\begin{figure}[H]
    \centering
    \begin{minipage}[t]{0.5\textwidth}
        \includegraphics[width=\textwidth]{images/s1mayor.png}
    \end{minipage}%
    \begin{minipage}[t]{0.5\textwidth}
        \includegraphics[width=\textwidth]{images/s1menor.png}   \end{minipage}%
    \caption{Tiempos de ejecucion vs largo de las cadenas}
    \label{fig:Tiempos de ejecucion vs largo de las cadenas}
\end{figure}


\newpage
\section{Conclusiones}
Tras el diseño, analisis y experimentación de los algoritmos de fuerza bruta y programación dinámica, se pueden concluir varios 
puntos importantes de resaltar, como que la complejidad temporal de la fuerza bruta es exponecial, lo que se traduce en un 
tiempo de ejecucion muy alto para grandes volumenes de datos, lo que se puede observar en los graficos proporcionados, por 
otro lado tenemos que la programación dinámica se comporta de manera mucho mas eficiente para volumenes muy grandes de datos,
teniendo estas 2 caracteristicas en cuenta, se puede concluir que la programación dinámica es la mejor opcion para resolver
este tipo de problemas con grandes volumenes de datos, o sea, para palabras muy largas (llevado a la vida real), sin embargo
se pudo ver que para las palabras vacias y/o una vacia y la otra no, la fuerza bruta es mas eficiente que la programación dinámica,
por lo que se tiene que programación dinámica es la mejor opcion para palabras de gran tamano ( > 10 caractares) y la fuerza bruta
es la mejor opcion para comparaciones donde una de las palabras es vacia o ambas palabras son vacias.Por otro lado
se tiene que, si se busca utulizar un algortimo que no consuma tanta memoria (o nula) es altamente recomendable la fuerza bruta
ya que como y hemos visto, la programación dinámica tiene una complejidad espacial mas alta que la ya dicha.Tambien se puede observar 
que se conserva la verasidad de los algoritmos ya que ambos algoritmos fueron capaces de encontrar una solucion a los diversos 
escenarios planteados.


La investigación de este problema nos permitió conocer y entender la importancia de los distintos algoritmos en la resolución del
problema de la distancia de edición, tambien quedo claro la importancia de seleccionar el algoritmo correcto segun las necesidades
del problema a resolver.

\newpage

\section{Condiciones de entrega}
\input{condiciones}

\newpage
\appendix



\input{sections/appendix1}
\printbibliography

\end{document}


