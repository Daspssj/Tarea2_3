Un resumen es un breve compendio que sintetiza todas las secciones clave de un trabajo de investigación: la introducción, los objetivos, la infraestructura y métodos, los resultados y la conclusión. Su objetivo es ofrecer una visión general del estudio, destacando la novedad o relevancia del mismo, y en algunos casos, plantear preguntas para futuras investigaciones. El resumen debe cubrir todos los aspectos importantes del estudio para que el lector pueda decidir rápidamente si el artículo es de su interés.

En términos simples, el resumen es como el menú de un restaurante que ofrece una descripción general de todos los platos disponibles. Al leerlo, el lector puede hacerse una idea de lo que el trabajo de investigación tiene para ofrecer \cite{elsevier_abstract_2024}.

\textbf{La extensión del resumen, para esta entrega, debe ser tal que la totalidad del índice siga apareciendo en la primera página. Recuerde que NO puede modificar el tamaño de letra, interlineado, márgenes, etc.}
