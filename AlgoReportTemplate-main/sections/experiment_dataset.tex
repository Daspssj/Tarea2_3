Los datasets utilizados para las pruebas fueron 6, y son de 20 palabras de largo variable
\begin{itemize}
    \item \textbf{dataset1:} Contiene palabras de misma longitud que varia entra 1 y 14 caracteres
    \item \textbf{dataset2:} Contiene palabras donde S1 es vacia y S2 tiene longitud aleatoria
    \item \textbf{dataset3:} Contiene palabras donde S2 es vacia y S1 tiene longitud aleatoria
    \item \textbf{dataset4:} Contiene palabras donde S1 es mayor o igual que S2
    \item \textbf{dataset5:} Contiene palabras donde S2 es mayor o igual que S1
    \item \textbf{dataset6:} Contiene palabras transpuestas
\end{itemize}

La importancia de estos datasets radica en que se pueden probar los algoritmos con distintos casos de prueba, y se pueden
verificar si los algoritmos son capaces de resolverlos de manera correcta en varios casos, esto nos puede dar una idea de
la eficiencia y efectividad de los algoritmos. mendiantes las pruebas que veremos acontinuacion.