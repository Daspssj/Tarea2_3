Tras el diseño, analisis y experimentación de los algoritmos de fuerza bruta y programación dinámica, se pueden concluir varios 
puntos importantes de resaltar, como que la complejidad temporal de la fuerza bruta es exponecial, lo que se traduce en un 
tiempo de ejecucion muy alto para grandes volumenes de datos, lo que se puede observar en los graficos proporcionados, por 
otro lado tenemos que la programación dinámica se comporta de manera mucho mas eficiente para volumenes muy grandes de datos,
teniendo estas 2 caracteristicas en cuenta, se puede concluir que la programación dinámica es la mejor opcion para resolver
este tipo de problemas con grandes volumenes de datos, o sea, para palabras muy largas (llevado a la vida real), sin embargo
se pudo ver que para las palabras vacias y/o una vacia y la otra no, la fuerza bruta es mas eficiente que la programación dinámica,
por lo que se tiene que programación dinámica es la mejor opcion para palabras de gran tamano ( > 10 caractares) y la fuerza bruta
es la mejor opcion para comparaciones donde una de las palabras es vacia o ambas palabras son vacias.Por otro lado
se tiene que, si se busca utulizar un algortimo que no consuma tanta memoria (o nula) es altamente recomendable la fuerza bruta
ya que como y hemos visto, la programación dinámica tiene una complejidad espacial mas alta que la ya dicha.Tambien se puede observar 
que se conserva la verasidad de los algoritmos ya que ambos algoritmos fueron capaces de encontrar una solucion a los diversos 
escenarios planteados.


La investigación de este problema nos permitió conocer y entender la importancia de los distintos algoritmos en la resolución del
problema de la distancia de edición, tambien quedo claro la importancia de seleccionar el algoritmo correcto segun las necesidades
del problema a resolver.