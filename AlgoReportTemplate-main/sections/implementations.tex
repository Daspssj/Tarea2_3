\begin{itemize}
    \item \textbf{Estructura de archivos:} El archivo donde se encuentra las implementacines de los algoritmos es
    Algoritmos.cpp , en este archivo se encuentran las implementaciones de los algoritmos de fuerza bruta y programación dinámica,
    como la funcion del main que se encarga de leer todos los datos de entrada (esto se refiere a las tablas de costos de 1 y 2 
    dimensiones junto con los inputs de los datasets que contienen los 2 strings a comparar y editar) y llamar a los algoritmos 
    correspondientes.


    En esta carpeta (Tarea 2 y 3) tambien se encuentran los costos de las operaciones de inserción, eliminación, 
    sustitución y transposición, como tambien las distintas datasets que se utilizaron para probar los algoritmos.


    Tambien se encuentra el archivo de Creardataset.cpp que se encarga de crear los datasets que se utilizaron para probar
    los algoritmos.
    \item \textbf{Funciones:} Las funciones mas importantes son dynamic\_progra() y brute\_force(), que corresponden a 
    las implementaciones de los algoritmos de programación dinámica y fuerza bruta, respectivamente.
\end{itemize}
