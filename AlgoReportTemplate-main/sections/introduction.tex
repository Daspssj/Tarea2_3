El problema de edicion de caracteres es un clasico problema en ciencias de la computacion, en el cual se busca encontrar 
la cantidad minima de operaciones para transformar una cadena de caracteres en otra. Este problema es de gran importancia en la vida cotidiana,
ya que se puede aplicar en la corrección de errores de texto, en la comparación de archivos, analisis de similitud de texto, etc.


Este informe esta enfocado en la implementación de 2 algoritmos para la resolución de estos problemas. En particular, 
se busca encontrar la solucion de, dada 2 cadenas de caracteres cual es el costo minimo de edicion de 1 de estas cadenas para que las cadenas sean iguales.
La investigacion de centra en analizar los distintos algoritmos para poder determinar cual es el mas eficiente en terminos de tiempo y espacio. Acaso
programación dinámica es mejor que fuerza bruta en todos los casos ? o hay casos en los que fuerza bruta es mejor que programación dinámica ?
Estas son algunas de las preguntas que se buscan responder con esta investigacion, la respuesta de estas preguntas es de gran importancia ya que
nos permitira saber como se comportan estos algoritmos en diferentes situaciones y asi poder elegir el algoritmo mas adecuado para cada caso.


Los algoritmos a implementar son, de programación dinámica y de fuerza bruta.


Las operaciones permitidas para este estudio seran: inserción, eliminación, sustitucion y transposicion de caracteres. Para la implementación 
de los algoritmos se utilizó el lenguaje de programación de C++.


Es importante que tengan a mano el codigo fuente de este estudio para poder seguir la explicación de los algoritmos, poder entender
como se implementan, como funcionan y por si se les presenta dudas sobre codigo en particular.
\url{https://github.com/Daspssj/Tarea2_3}
